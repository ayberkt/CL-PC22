\documentclass[handout,11pt]{beamer}
%\documentclass[slides]{beamer}

%\usetheme{Warsaw}
\usetheme{Montpellier}
\usefonttheme[onlymath]{serif}
\setbeamertemplate{headline}{}
\setbeamertemplate{navigation symbols}{}

%\usepackage[english,norsk,german]{babel}
\usepackage[utf8x]{inputenc}
\usepackage{times}
\usepackage[T1]{fontenc}
\usepackage{proof}
\usepackage{url}
\usepackage{verbatim}
\usepackage{hyperref}
\usepackage{graphics}

\usepackage{color}
\newcommand{\blue}[1]{\textcolor{blue}{#1}}
\newcommand{\red}[1]{\textcolor{red}{#1}}
\newcommand{\green}[1]{\textcolor{green}{#1}}

%\usepackage[all]{xy}
\usepackage{amssymb,amsmath}
%\usepackage{stmaryrd}

\newtheorem{proposition}{Proposition}

\newcommand{\weg}[1]{}
\newcommand{\Prop}{\mathit{Prop}}
\newcommand{\nat}{\mathbb{N}}
\newcommand{\set}[1]{\{#1\}}
\newcommand{\liff}{\leftrightarrow}
\newcommand{\proves}{\vdash}
\newcommand{\forces}{\mathrel{\Vdash}}
\newcommand{\deca}{${\lor}{\exists}{\land}$}

\newcommand{\XX}{\mbox{\sf X}}
\newcommand{\ndE}{\vdash_{\es}}
\newcommand{\ndX}{\vdash_{X}}
\newcommand{\ndxx}{\vdash_{\vec x}}
\newcommand{\ndY}{\vdash_{Y}}
\newcommand{\ndXx}{\vdash_{X,x}}
\newcommand{\ndXxi}{\vdash_{X,\vec x_i}}
\newcommand{\es}{\emptyset}

\newcommand{\covd}{\mathrel{\vartriangleleft}}
\newcommand{\covs}{\mathrel{\vartriangleright}}
\newcommand{\covT}{\mathrel{\vartriangleleft_T}}

\newcommand*{\C}{{\mathbb{C}}}
\newcommand*{\D}{{\mathbb{D}}}
\newcommand*{\E}{{\mathcal{E}}}
\newcommand*{\subvs}{_{\mathrm{vs}}}
\newcommand*{\subrn}{_{\mathrm{rn}}}
\newcommand*{\subts}{_{\mathrm{ts}}}

\newcommand{\Cvs}{\mathbb{C}\subvs}
\newcommand{\Crn}{\mathbb{C}\subrn}
\newcommand{\Cts}{\mathbb{C}\subts}
\newcommand{\Tvs}{T\subvs}
\newcommand{\Trn}{T\subrn}
\newcommand{\Tts}{T\subts}
\newcommand{\Fts}{\Vdash\subts}
\newcommand{\Fvs}{\Vdash\subvs}
\newcommand{\Frn}{\Vdash\subrn}
\newcommand{\FTvs}{\Vdash^T\subvs}
\newcommand{\FTrn}{\Vdash^T\subrn}
\newcommand{\stdcat}[1]{\textup{\bfseries #1}}
\newcommand{\blank}{\textup{--}}


\DeclareMathOperator{\Id}{Id}
\DeclareMathOperator{\id}{id}
\DeclareMathOperator{\Hom}{Hom}
\DeclareMathOperator{\ConFilt}{ConFilt}
\DeclareMathOperator{\Sh}{Sh}
\DeclareMathOperator{\Psh}{PSh}
\DeclareMathOperator{\Ob}{Ob}
\DeclareMathOperator{\dom}{dom}
\DeclareMathOperator{\cod}{cod}
\DeclareMathOperator{\Tm}{Tm}
\DeclareMathOperator{\Fact}{Fact}
\DeclareMathOperator{\Ima}{Im}
\DeclareMathOperator{\NV}{NV}



\title{A Site Model for Coherent Logic}
\author{Marc Bezem\\
Department of Informatics\\
University of Bergen\\
jww\\
U. Buchholtz and T. Coquand, 2018
}
\date{September 2022}

\begin{document}

\frame{\titlepage}

\section[Outline]{}
\frame{\tableofcontents}


\section{A category of forcing conditions}

\subsection{States as a category}

\begin{frame}
\frametitle{Coherent logic preliminaries}
 \begin{itemize}[<+->]   %  [<+-| alert@+>]
  \item Sources: 
   \begin{itemize}[<+->]   %  [<+-| alert@+>]
    \item Th{\'e}ories coh{\'e}rentes et topos coh{\'e}rents,
    M.-F.~Coste and M.~Coste, 1975.
    \item Syntactic Forcing Models for Coherent Logic, 
    B. and U. Buchholtz and T. Coquand, 2018
    \end{itemize}
   \item Fix a finite first-order signature $\Sigma$
   \item Fix a countably infinite set of variables $\XX=\set{x_0,x_1,\ldots}$
   \item Let $\Tm(X)$ be the set of $\Sigma$-terms over $X \subseteq \XX$    
   \item Define the category $\Cts$ having:
    \begin{itemize}[<+->]   %  [<+-| alert@+>]
    \item Objects denoted as pairs $(X;A)$,
    where $X$ is a finite subset of $\XX$ and $A$ is a finite set
    of atoms in the language defined by $\Sigma$ and $X$.
    (This means that only variables from $X$ may occur in $A$.)
    Such pairs $(X;A)$ are called \emph{conditions}.
    \item Morphisms denoted as $f : (Y;B) \to (X;A)$,
where $f$ is a \emph{term substitution} $X \to \Tm(Y)$ such that $Af \subseteq B$
    \end{itemize}
 \end{itemize}
\end{frame}


\begin{frame}
\frametitle{CL as fragment of FOL}
 \begin{itemize}[<+->]   %  [<+-| alert@+>]
    \item Closed formulas of the form $\alert{\forall\vec{x}.}~C \to D$ \alert{(scope!)}, where ...
    \item $C$ a conjunction of atoms $A_1 \land \cdots \land A_n~(n\geq 0)$
    \item $D$ a disjunction $E_1 \lor \cdots \lor E_m~(m\geq 0)$, where ...
    \item each $E_i$ is a formula of the form $\exists \vec{y}.~B_1 \land \cdots \land B_k~(k\geq 0)$
              where each $B_j$ is an atom ($\vec{y}, B_j, k$ may depend on $i$)
    \item Short forms, implicitly universally closed: 
    \begin{itemize}
       \item $\top$ for empty conjunction, $\bot$ for empty disjunction         
       \item $E$ (as above) for $\forall \vec x.~\top\to E$
       \item $(A_1 \land \cdots \land A_n)\to(B_1 \lor \cdots \lor B_m)$, resolution clauses
       \item $(A_1 \land \cdots \land A_n)\to \exists y.~B$, one existential quantifier
    \end{itemize}
 \end{itemize}
\end{frame}

\begin{frame}
\frametitle{History of CL}
 \begin{itemize}[<+->]   %  [<+-| alert@+>]
    \item Skolem (20s):  lattice theory and projective geometry
    \item Grothendieck (70s): geometric morphisms preserve geometric
              logic (= coherent logic + infinitary disjunction)
    %\item Janicic, Kordic (90s): geometry prover Euclid
    %\item Abdennadher, Sch\"utz (90s): \alert CPUHR (\alert Constraints)
    %\item B, Coquand, Fisher, Nivelle (00s): geometric/coh.\ logic
    %\item De Roo (00s): Euler (EYE, N3, Semantic Web)
 \end{itemize}
\end{frame}

\begin{frame}
\frametitle{Examples}
 \begin{itemize}[<+->]   %  [<+-| alert@+>]
    \item lattice theory: $%\forall x,y.~
    \exists z.~\mathit{meet}(x,y,z)$
    \item geometry: $%\forall x,y.~
    p(x)\land p(y) \to \exists z.~\ell(z) \land i(x,z) \land i(y,z)$
    \item rewriting: $%\forall x,y,z.~
    r(x,y)\land r(x,z) \to \exists u.~r(y,u)\land r(z,u)$
    \item fol-tc: $%\forall x,y.~
    r^*(x,y)\to (x=y)\lor\exists z.~r(x,z)\land r^*(z,y)$     
    \item seriality: $%\forall x.~
    \exists y.~s(x,y)$ (who needs a function?)
    \item field theory: $%\forall x.~
    (x=0) \lor \exists y.~(x\cdot y=1)$
    \item non-empty domain: $\exists x.~\top$
%    \item non-empty domain: $\exists x.~\top$
 \end{itemize}
\end{frame}

\begin{frame}
\frametitle{A proof theory for CL}
 \begin{itemize}[<+->]   %  [<+-| alert@+>]
    \item Ground forward chaining with case distinction and introduction of witnesses
(ground tableau reasoning)
    \item Inductive definition of $\set{F_0,\ldots,F_{n-1}}\vdash_\Gamma F$ possible %(see papers)
    \item Example: infer $r$ from $p\lor \exists x.~q(x),~p\to\bot,~q(y)\to r$
    \[
\infer[p\vee \exists x.~q(x)]{\emptyset}
{
\infer[p\to\bot]{\set{p}}{(\bot)} & \infer[q(y)\to r]{\set{q(c)}}{\set{q(c),r}}
}
\]
    \item Close branch if conclusion has been inferred,
    or $\bot$ (zero cases); stop if no new info can be inferred (model found!) 
    \item Proof steps are simple and intuitive (but can be many)
    \item Easy conversion of tableau to natural deduction, lambda term, or stylized natural language
 \end{itemize}
\end{frame}


\end{document}
