\documentclass{article}

\usepackage{a4wide}

\begin{document}

\begin{center}
\textbf{Manual for use of the CL prover cl.pl in Prolog}
\end{center}

\paragraph{General instructions}

\begin{enumerate}
\item Install SWI-Prolog from \texttt{www.swi-prolog.org} or distro.
Currently version 8.0.3 on Mac OS and 7.3.4 on Linux Mint.
\item The command \texttt{swipl} launches SWI-Prolog
enter \texttt{[cl].}after the \texttt{?-} prompt to load the CL prover.
\item There are basically five commands that you can use after the
  \texttt{?-} prompt:
\begin{itemize}
\item \texttt{test(<file>)} tries to prove the proposition goal from
  the axioms in \texttt{<file>.in}.
\item \texttt{out(<file>)} does the same as \texttt{test(<file>)} but
  writes to \texttt{<file>.out}.
\item \texttt{prf(<file>)} reconstructs a proof from the log and
  writes this to \texttt{<file>.prf}.
\item \texttt{coq(<file>)} generates a Coq script from
  \texttt{<file>.prf} and writes to \texttt{<file>.v}.
\item \texttt{run(<file>)} does \texttt{out}, \texttt{prf} and
  \texttt{coq} with \texttt{<file>} and lists axioms that are not used.
\end{itemize}
\item Try \texttt{run(dpe)} to make sure everything is fine.   
\item If you have Coq \texttt{coq.inria.fr} installed, 
  you can check the generated proofs by the command \texttt{coqc <file>.v}
  after the Unix prompt. This creates \texttt{<file>.vo} which could
  be used in later Coq developments.
\end{enumerate}


\paragraph{Instructions for preparing input files}

The prover always tries to find a proof of \texttt{goal}
\begin{enumerate}
\item If you want to prove $p \wedge q$, the first axiom should be:
\begin{verbatim}
   _ axiom <name>:(p,q => goal).
\end{verbatim}
\item If you want to prove $p \vee q$, the first two axioms should be:
\begin{verbatim}
   _ axiom <name1>: (p => goal).
   _ axiom <name2>: (q => goal).
\end{verbatim}
 % Proving that a theory is inconsistent goes by proving goal WITHOUT
 % axioms as above.
\item If you want to prove that there exists an $x$ such that $p(x)$,
  the first axiom should be:
\begin{verbatim} 
   _ axiom <name>(X): (p(X) => goal).
\end{verbatim}
   Note that in Prolog variables start with an upper case letter.

\item If you want to prove that, e.g., for all $x$ with $p(x)$ there exists
   a $y$ such that $q(x,y)$ the first three lines should be:
\begin{verbatim}
   dom(a). % declaration of a new constant
   _ axiom <name1>: (true => p(a)).
   _ axiom <name2>(Y): (q(a,Y) => goal).
 \end{verbatim}
 \item For proving an arbitrary coherent formula you must craft some
   combination of 1-4.
 \item Recall that the search is depth-first so that a reasonable
   ordering is: first the Horn clauses, then the disjunctive clauses
   and finally axioms like seriality.  This will work in many cases.
\item The general format of an input clause is:
\begin{verbatim}
   <control> axiom <name>(<variables>): (<conj. of atoms> => <conclusion>).
\end{verbatim}
  The \texttt{<variables>} are the universal variables of the axiom.
  Use Prolog's disjunction \texttt{;}. Universal variables must occur
  in the premiss (range restriction), if necessary use the domain
  predicate \texttt{dom}. For example, reflexivity should NOT be
\begin{verbatim}
   _ axiom refl(X): (true => r(X,X)) 
\end{verbatim}
but 
\begin{verbatim}
   _ axiom refl(X): (dom(X) => r(X,X)). 
\end{verbatim}
Right of \texttt{=>} the domain predicate is used to express
existential quantification.  Thus, e.g., for all $x$ we have $p$ or
there exists a $y$ such that $q(x,y)$ should be:
\begin{verbatim}
   _ axiom <name>(X): (dom(X) => p ; dom(Y),q(X,Y)).
\end{verbatim}
\item Predicates you use should be declared with their arity, e.g.,
\begin{verbatim}
   :- dynamic p/0,q/2.
\end{verbatim}
\end{enumerate}


\paragraph{The pragmatics of the CL prover}

\begin{enumerate}
\item For small problems the following Law of Nature for AR seems to
  be valid: Either the problem is solved in 100 milliseconds, or it is
  not solved at all.
\item Of a complete run, test and coq take each 10\% of the CPU time
  and prf the rest.
\item (OPTIONAL) Strategies can be programmed with the binary predicate
  \texttt{enabled(<control>,<guard>)} and the ternary predicate
  \texttt{next(<control>,<guard>,<new\_guard>)}. This requires some
  skill. It may cause false negatives and even runtime errors. False
  positives have not happened in practice, but could happen in
  principle, f.e., after \texttt{assert(false)} when the search has
  started.  False positives NEVER EVER pass the typecheck of Coq. :-)
 \item Launch \texttt{grep enabled *.in} to see which examples use
   strategies and how. Never use a strategy when you do not understand
   why the CL prover doesn't find the proof.
\item Ignore warnings about redefined static procedures.
\end{enumerate}

\paragraph{Suggestions for small experiments}

\begin{enumerate}
\item Run \texttt{or.in}, \texttt{exist.in} and \texttt{drinker.in}
  and understand the \texttt{.out} files completely.

\item Prove $p$ in the coherent theory $(\exists x\, q(x)) \vee p,\, q(x)
  \to p$.
  
\item Prove that a transitive and symmetric relation $r$ satisfies
  $r(x,y) \to r(x,x)$.
  
\item The file \texttt{dpe.in} expresses that the diamond property is
  preserved under reflexive closure of the rewrite relation, that is,
  if $r$ satisfies the axiom $\mathit{dp\_r}$ then $\mathit{re}$
  defined by $\mathit{e\_in\_re}$, $\mathit{r\_in\_re}$,
  $\mathit{e\_or\_r}$ also satisfies the diamond property.  Run
  \texttt{dpe.in} and understand the \texttt{.out} file completely.
  Explain why the proof is far from optimal and give a better proof by
  human intelligence.
  
\item Prove that a relation $r$ which satisfies the diamond property
  is serial: $r(x,y) \to (\exists z\, r(y,z))$.
  
\item Translate to CL and prove: $(\forall x (\exists y (p(x) \vee q(y))))
  \to (\exists y (\forall x (p(x) \vee q(y))))$.
  
\item Let $\mathit{lt}$ be transitive and strongly serial: 
$\forall x~\exists y\, lt(x,y))$. Show
  by translation to CL that $p(x) \leftrightarrow \forall y
  (\mathit{lt}(x,y) \to \neg p(y))$ leads to a contradiction if the
  domain is non-empty.

\end{enumerate}   


\end{document}
