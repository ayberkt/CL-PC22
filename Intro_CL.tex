\documentclass[handout,11pt]{beamer}
%\documentclass[slides]{beamer}

%\usetheme{Warsaw}
\usetheme{Montpellier}
\usefonttheme[onlymath]{serif}
\setbeamertemplate{headline}{}
\setbeamertemplate{navigation symbols}{}

%\usepackage[english,norsk,german]{babel}
\usepackage[utf8x]{inputenc}
\usepackage{times}
\usepackage[T1]{fontenc}
\usepackage{proof}
\usepackage{url}
\usepackage{verbatim}
\usepackage{hyperref}
\usepackage{graphics}

\usepackage{color}
\newcommand{\blue}[1]{\textcolor{blue}{#1}}
\newcommand{\red}[1]{\textcolor{red}{#1}}

%\usepackage[all]{xy}
\usepackage{amssymb,amsmath}
%\usepackage{stmaryrd}

\newtheorem{proposition}{Proposition}

\newcommand{\weg}[1]{}
\newcommand{\Prop}{\mathit{Prop}}
\newcommand{\nat}{\mathbb{N}}
\newcommand{\set}[1]{\{#1\}}
\newcommand{\imp}{\rightarrow}
\newcommand{\liff}{\leftrightarrow}
\newcommand{\proves}{\vdash}
\newcommand{\deca}{${\lor}{\exists}{\land}$}

\title{Coherent Logic --- an overview}
\author{Marc Bezem\\
Department of Informatics\\
University of Bergen
}
\date{September 2022}

\begin{document}

\frame{\titlepage}

\section[Outline]{}
\frame{\tableofcontents}


\section{Crash course in CL}

\subsection{Basics}

\begin{frame}
\frametitle{Coherent logic preliminaries 1}
 \begin{itemize}[<+->]   %  [<+-| alert@+>]
    \item Fix a first-order signature $\Sigma$
    \item Positive formulas: built up from atoms using 
    $\top,\bot,\lor,\land,\exists$
    \item Coherent implications (sentences): $\forall\vec{x}.~(C \to D)$ 
    with $C,D$ positive formulas
    \item Coherent theory: axiomatized by coherent sentences
    \item \deca-formula: 
    $(\exists\vec{y}_1.A_1) \lor \cdots \lor (\exists\vec{y}_k.A_k)$,
    $k\geq 0$, with each $A_i$ a (possibly empty) conjunction of atoms
    \item Theorem 1. Every positive formula is (constructively) equivalent to
    a \deca-formula. Proof by induction:
    \begin{itemize}[<+->]   %  [<+-| alert@+>]
    \item Base cases: atom (one disjunct, empty $\exists$, one conjunct);\\ 
    $\bot$ (empty $\lor$); 
    $\top$ (one disjunct with empty $\exists,\land$)
    \item Induction cases: $\lor$ (trivial); 
    $\land$ (distributivity + $(\exists x.\varphi) \land (\exists y.\psi)$ iff 
                              $\exists xy.~(\varphi\land\psi)$);
    $\exists$ (commutes with $\vee$)    
    \end{itemize}
 \end{itemize}
\end{frame}

\begin{frame}
\frametitle{Coherent logic preliminaries 2}
 \begin{itemize}[<+->]   %  [<+-| alert@+>]
    \item Theorem 2. Every coherent implication is (constructively) 
    equivalent to a finite set of coherent implications
    $\forall\vec{x}.~(C \to D)$ 
    with $C$ a conjunction of atoms and $D$ a \deca-formula
    \item Proof. Use Thm.1 to replace $C$ and $D$ by \deca-formulas.
    Then use $(\varphi\lor\psi)\to D$ iff $(\varphi\to D)\land(\psi\to D)$,
    and $(\exists y.\varphi)\to D$ iff $\forall y.~(\varphi\to D)$
    \item Notation: we use the format of Theorem 2, 
    leaving out the universal prefix, and omitting the premiss $C\to{}$ 
    if $C\equiv\top$
    \item Discuss: $\exists y.\top$ and $\exists y.\bot$ and
    $\forall y.\top$ and $\forall y.\bot$
    \item Full compliance with Tarski semantics if $\Sigma$ has a constant
 \end{itemize}
\end{frame}

\begin{frame}
\frametitle{Examples}
 \begin{itemize}[<+->]   %  [<+-| alert@+>]
    \item all usual equality axioms, including congruence
    \item $p\lor np$ and $p\land np\to\bot$ 
    (NB $p\lor\neg p$ is \red{not} coherent)
    \item lattice theory: $%\forall x,y.~
    \exists z.~\mathit{meet}(x,y,z)$
    \item geometry: $%\forall x,y.~
    p(x)\land p(y) \to \exists z.~\ell(z) \land i(x,z) \land i(y,z)$
    \item rewriting: $%\forall x,y,z.~
    r(x,y)\land r(x,z) \to \exists u.~r(y,u)\land r(z,u)$
    \item weak-tc-elim: $%\forall x,y.~
    r^*(x,y)\to (x=y)\lor\exists z.~r(x,z)\land r^*(z,y)$     
    \item seriality: $%\forall x.~
    \exists y.~s(x,y)$ (who needs a function?)
    \item field theory: $%\forall x.~
    (x=0) \lor \exists y.~(x\cdot y=1)$
    \item local ring: 
    $\exists y.~(x\cdot y = 1) \lor (\exists y.~((1-x)\cdot y = 1)$
 \end{itemize}
\end{frame}


\begin{frame}
\frametitle{History of CL}
 \begin{itemize}[<+->]   %  [<+-| alert@+>]
    \item Skolem (1920s): coherent formulations of lattice theory 
    and projective geometry, calling the axioms ``Erzeugungsprinzipien"
    (production rules), anticipating ground forward reasoning. Using CL:
    \begin{itemize}[<+->]   %  [<+-| alert@+>]
    \item Skolem solved a decision problem in lattice theory
    \item Skolem gave a method to test dependence from the axioms
    of plane projective geometry   
    \end{itemize}
    
    \item Grothendieck (1960s): geometric morphisms preserve geometric
              logic (= coherent logic + infinitary disjunction).
    This is quite complicated, but we'll see a glimpse of the (related)
    forcing semantics later.
    %\item Janicic, Kordic (90s): geometry prover Euclid
    %\item Abdennadher, Sch\"utz (90s): \alert CPUHR (\alert Constraints)
    %\item B, Coquand, Fisher, Nivelle (00s): geometric/coh.\ logic
    %\item De Roo (00s): Euler (EYE, N3, Semantic Web)
 \end{itemize}
\end{frame}


\begin{frame}
\frametitle{A proof theory for CL}
 \begin{itemize}[<+->]   %  [<+-| alert@+>]
    \item Ground forward chaining with case distinction and introduction of witnesses
(ground tableau reasoning)
    \item Inductive definition of $\set{F_0,\ldots,F_{n-1}}\vdash_\Gamma F$ possible %(see papers)
    \item Example: infer $r$ from $p\lor \exists x.~q(x),~p\to\bot,~q(y)\to r$
    \[
\infer[p\vee \exists x.~q(x)]{\emptyset}
{
\infer[p\to\bot]{\set{p}}{(\bot)} & \infer[q(y)\to r]{\set{q(c)}}{\set{q(c),r}}
}
\]
    \item Close branch if conclusion has been inferred,
    or $\bot$ (zero cases); stop if no new info can be inferred (model found!) 
    \item Proof steps are simple and intuitive (but can be many)
    \item Easy conversion of tableau to natural deduction, lambda term, or stylized natural language
 \end{itemize}
\end{frame}

\subsection{Metatheory}

\begin{frame}
\frametitle{Meta (Tarski semantics)}
 \begin{itemize}[<+->]   %  [<+-| alert@+>]
    \item Soundness easily proved by induction (constructive)
    \item Refutation completeness: assume $T$ has no models. Do forward reasoning in
an exhaustive and fair way. Every infinite branch would constitute
a model (why? fair!), which is impossible. By K\"onig's Lemma the tree must be finite,
with all finite branches closed by $\bot$ (no models!). Hence the tree is a proof of 
$\bot$.  
    \item Completeness for arbitrary facts: very similar
    \item Completeness for arbitrary coherent formulas: ... (explain)
 \end{itemize}
\end{frame}

\begin{frame}
\frametitle{Meta (ctnd)}
 \begin{itemize}[<+->]   %  [<+-| alert@+>]
    \item Corollary of completeness: 
    all classically provable coherent sentences are constructively provable
    \item For \alert{geometric} logic this was discovered by Barr
    \item Completeness and Barr's Theorem \alert{not constructive}
    \item Coherent completeness wrt forcing semantics is constructively
         provable, but does not give the conservativity of classical reasoning
    \item NB: the forcing semantics is sound wrt intuitionistic logic for arbitrary formulas
 \end{itemize}
\end{frame}


\begin{frame}
\frametitle{Rationale for automated reasoning}
 \begin{itemize}[<+->] %[<+->]   %  [<+-| alert@+>]
    \item Expressivity of CL is between CNF (clauses) and FOL
    \item Different balance: expressivity versus efficiency
    \item Skolemization not necessary
    \begin{itemize}
       \item Skolemization makes the Herbrand Universe infinite
       \item Why skolemize an axiom like $p(x,y)\to\exists z.~p(x,z)$?
       \item Skolemization changes meaning
       \item Skolem functions obfuscate symmetries
       \item \alert{But: skolemized proofs can be exponentially shorter!}
    \end{itemize}
 \end{itemize}
\end{frame}

\subsection{Translation from FOL to CL}

\begin{frame}
\frametitle{Translation from FOL to CL}
 \begin{itemize}[<+->]
   \item NB: $\forall x.\,\varphi \to\bot$ only coherent for \alert{conjunctions of atoms} $\varphi$
   \item A good start (but not for constructivists!): NNF
   \item THM: Every FOL theory has a definitional extension that is equivalent to a CL theory
   \item Proof: introduce two new predicates $T(\varphi),F(\varphi)$ for each subformula $\varphi$, with as parameters the free variables of $\varphi$.
 Use the rules for signed tableaux as coherent definitions (see next).
Add $T(\psi)$ for every $\psi$ in the theory
 \item Base cases: $\forall\vec x.\,T(p) \lor F(p)$, $\forall\vec x.\,T(p) \land F(p) \imp \bot$
(\alert{!})
\item Propositional connectives: as expected (but a lot of variation possible)
 \end{itemize}
\end{frame}
 

\begin{frame}
\frametitle{From FOL to CL (ctnd)}
 \begin{itemize}[<+->]
\item Make coherent, using the smaller subformulas $T(\varphi),F(\varphi)$:
 \begin{itemize}
 \item $T(\forall x.~\varphi) \liff \neg\exists x.~F(\varphi)$
 \item $F(\forall x.~\varphi) \liff \exists x.~F(\varphi)$
 \end{itemize}
\item Coherent formulas:
 \begin{itemize}
 \item $\forall x.~(T(\forall x.~\varphi) \to T(\varphi))$
 \item $T(\forall x.~\varphi) \lor \exists x.~F(\varphi)$
 \item $F(\forall x.~\varphi) \to \exists x.~F(\varphi)$
 \item $\forall x.~(F(\varphi) \to F(\forall x.~\varphi))$
 \end{itemize}
\item Similarly for $T(\exists x.~\varphi),F(\exists x.~\varphi)$
\item Optimization desirable and possible (Polonsky, PhD'11)
\item Proof recovery: $T:= \lambda\varphi.\,\varphi$, $F:= \lambda\varphi.\,\neg\varphi$
\item Translations steps are easily (classically) provable
\end{itemize}
\end{frame}
 
\begin{frame}
\frametitle{Example: proving Peirce's Law}
 \begin{itemize}[<+->] 
    \item Peirce's Law: $((p\imp q)\imp p)\imp p$
    \item Coherent theory:
    \begin{enumerate}[<+->]
    	\item $F(((p\imp q)\imp p)\imp p)$
    	\item $F(((p\imp q)\imp p)\imp p) \imp T((p\imp q)\imp p) \wedge F(p)$
    	\item $T((p\imp q)\imp p) \imp F(p\imp q) \vee T(p)$
    	\item $F(p\imp q) \imp T(p) \wedge F(q)$
    	\item $T(p) \wedge F(p) \imp \bot$
    \end{enumerate}
    \item Proof: use 1,2,3 and split on $F(p\imp q) \vee T(p)$, ...
    \item Discussion: 
    \begin{itemize}[<+->]
      \item $T((p\imp q)\imp p) \imp F(p\imp q) \vee T(p)$ was a \emph{choice}
      \item Why not $T((p\imp q)\imp p) \wedge T(p\imp q) \imp T(p)$ ?
      \item Then you would need $F(p\imp q) \vee T(p\imp q)$ ...
      \item Signed tableaux: positive and negative rules  %(not all needed here)
  %  \item Sometimes no disjunctions needed: $p\imp q, p \proves q$
     \item Polonsky's translation looks for optimal polarities
 \end{itemize}  
 \end{itemize}
\end{frame}


\begin{frame}
\frametitle{FOL vs CL, conclusions}
 \begin{itemize}[<+->]
\item Consequences in CL are always constructive
\item Translation of FOL to CL contains many non-constructive steps,
seemingly more than necessary
\item Open problem: how to deal with \alert{intuitionistic} FOL tautologies
in an optimal way?
\item Moral: CL works best for theories that are (almost) coherent
\item And: life ain't easy for a constructivist!
\end{itemize}
\end{frame}

\subsection{Elimination of function symbols}

\begin{frame}
\frametitle{Elimination of function symbols}
 \begin{itemize}[<+->]
    \item Predicates for constants: $c(x)$ for $c=x$, axiom $\exists x.\,c(x)$
    \item Binary predicates for unary functions: $s(x,y)$ for $s(x)=y$, axiom $\exists y.\,s(x,y)$
    \item Similarly: $(n{+}1)$-predicates for $n$-ary functions
    \item Unicity, e.g., $c(x)\land c(y)\to x=y$, \alert{not} required! (Why?)
    \item Example: e.g., $a=b$ becomes $a(x) \land b(y) \to x=y$
    \item Equality almost vanishes: only needed between variables
    \item Translation from FOL to CL eliminating functions: CUDEN (Nivelle and Meng, IJCAR 06)
 \end{itemize}
\end{frame}

\begin{frame}
\frametitle{Puzzle, formalized in CL with functions (Nivelle)} 
 \begin{itemize}[<+->]
     \item Can one color each
     $n\in\nat$ either red or green but not both such that, if $n$ is red
          then $n{+}2$ is green, and if  $n$ is green, then $n{+}1$ is red?
    \item No: consider $r?gr\ldots$~ and~ $gr?gr\ldots$
    \item CL theory:
    \begin{enumerate}
       \item $r(x) \vee g(x)$ 
       \item $r(x) \wedge g(x) \imp \bot$
       \item $r(x) \imp g(s(s(x)))$ 
       \item $g(x) \imp r(s(x))$
     \end{enumerate}
   \item Do we miss something?
   \item Yes, domain non-empty: 
    \begin{enumerate}       
       \setcounter{enumi}{4}       
       \item $\exists x.~\top$
    \end{enumerate}
 \end{itemize}
\end{frame}

\begin{frame}
\frametitle{Puzzle, functions eliminated}
 \begin{itemize}[<+->]
    \item Living dangerously: demo of {\tt hdn.co} in Colog (Fisher, '12)
    \item Just in case the demo fails:
    \begin{enumerate}
      \item $r(x) \vee g(x)$ 
      \item $r(x) \wedge g(x) \imp \bot$
      \item $r(x) \wedge s(x,y) \wedge s(y,z) \imp g(z)$ 
      \item $g(x) \wedge s(x,y) \imp r(y)$    
      \item $\exists x.~\top$
      \item $\exists y.\,s(x,y)$
  \end{enumerate}
  %\item Proof recovery
   \end{itemize}
\end{frame}

\begin{frame}
\frametitle{Proof procedure, recovery of proof object}
 \begin{itemize}[<+->]
    \item If necessary, translate FOL $\rightarrow$ CL
    \item Prove (refute) with prover
    \item Proofs are valid for all relations %and all $T,F$
    \item Substitute $s(x,y) := (s(x) = y)$, assume constant $0$
    \item $\exists y.\,s(x)=y$ and $\exists x.~\top$ easily provable
    \item $g(x) \wedge s(x)=y \imp r(y)$ is equivalent to $g(x) \imp r(s(x))$
    \item Similarly for all other axioms
    \item Proof (refutation) of original formula is obtained
 \end{itemize}
\end{frame}


\section{Applications}

\begin{frame}
\frametitle{Applications}
 \begin{itemize}[<+->]   %  [<+-| alert@+>]
    \item automated reasoning
    \item proof assistants    
    \item model finding
    \item constructive algebra
 %   \item constraint handling (\alert{maybe CL is application of CHR} :-)
 \end{itemize}
\end{frame}

\subsection{Automated reasoning}

\weg{
\begin{frame}
\frametitle{Example: DP closed under reflexive closure}
\small\verbatiminput{dpe.co}
\end{frame}

\frame
  {    
    \frametitle{Search space} \vspace*{-.95in}
    \scalebox{0.40}
      {
    \includegraphics{tree_dpe}
      }
  }
}%endweg

\begin{frame}
\frametitle{Automated reasoning}
 \begin{itemize}[<+->]   %  [<+-| alert@+>]
    \item Depth-first search: not complete (but quite good)
    \item Standard ordering of coherent formulas:
    \begin{enumerate}
       \item Facts first, then Horn clauses ($\to$ goal)
       \item Disjunctive clauses
       \item Existential clauses
       \item Disjunctive existential clauses
    \end{enumerate}
    \item Breadth-first search: complete (but often slow)
    \item Next slide: queueing depth-first
 \end{itemize}
\end{frame}

\begin{frame}
\frametitle{Queueing depth-first}
 \begin{itemize}[<+->]
    \item Obvious: $\exists y.\,s(x,y)$ harmful for depth-first search
    \item Without functions, depth-first terminates for forms 1 and 2
    \item Depth-first search not complete for \alert{one single existential} clause, subtle:
\texttt{  p(a). p(b). q(b) -> goal. p(X),p(Y) -> dom(U),p(U),q(X),r(Y).}
    \item Wanted: fair application of forms 3 and 4 
    \item Queueing depth-first: the (disjunctive) existential clauses
          in a cyclic queue $+$ iterative deepening wrt constants. Complete.
 \end{itemize}
\end{frame}

\begin{frame}
\frametitle{Automated reasoning in CL, conclusions}
 \begin{itemize}[<+->]
   \item Good start: Newman's Lemma (B, Coquand, BEATCS'03)
   \item Limited success in CASC: 50\% in FOF (Geo, Nivelle'06)
   \item Readable proofs can be extracted from search space
   \item Highlight: Hessenberg's Theorem (B, Hendriks, JAR'08)
   \item Promising: using CDCL techniques (Nikolic, PhD'13)
    \item Colog:\\\url{www.csupomona.edu/~jrfisher/colog2012/}
 \end{itemize}
\end{frame}

\subsection{Proof assistants}

\begin{frame}
\frametitle{Proof assistants}
 \begin{itemize}[<+->]   %  [<+-| alert@+>]
    \item Proof objects \alert{required}
    \item CL proofs are readable and easy to convert
    \item Provers outputting proof objects
    \begin{itemize}
       \item CL.pl (exports proofs to Coq)
       \item \url{coherent} (Isabelle tactic, Berghofer)
       \item ArgoCLP (Coq, Isar, `natural' language)
    \end{itemize}
    \item %\href{file://testscript.html}{\beamergotobutton{Excerpt}} 
                Excerpt: Stojanovic et al., CICM'14
                \href{run:/usr/bin/evince}{\beamergotobutton{evince}}
 \end{itemize}
\end{frame}

\subsection{Model finding}

\begin{frame}
\frametitle{Model finding}
 \begin{itemize}[<+->]   %  [<+-| alert@+>]
  %  \item Example: DP not closed under inverse of reflexive
    %          $r$ \hyperlink{nDPmodel}{\beamergotobutton{Go!}}
    \item Breadth-first not finite model complete: $\exists y.~s(x,y)$
    \item Solution (Nivelle \& Meng, IJCAR'06): try old constants
    before you generate a new one
    \item Complete, but lemma learning \alert{absolutely necessary}!
    \item Success in CASC'07: 81\% in FNT (Geo, Nivelle)\\
             (Paradox, based on Minisat, winner with 85\%)
    \item CL competitive on problems `too big to ground'
    \item Example: formal verification of a Kripke model for simplicial sets (B \& Coquand, TCS)
 \end{itemize}
\end{frame}              


\weg{
\begin{frame}[label=nDPmodel]
%\frametitle{Example: DP closed under reflexive closure}
\small\verbatiminput{ndp.co}
\end{frame}

\frame
  {    
    \frametitle{Model} \vspace*{-.95in}
    \scalebox{0.40}
      {
    \includegraphics{tree_dpe}
      }
  }
}%endweg

\subsection{Constructive algebra}

\begin{frame}
\frametitle{Constructive algebra}
 \begin{itemize}[<+->]   %  [<+-| alert@+>]
    \item Pioneers of applying CL/GL to constructve algebra:\\
  Coste, Lombardi, Roy, %(Effective Nullstellens\"atze ...), 
Coquand
    \item Idea: making constructive sense of classical proofs by
  exploiting that significant parts of algebra can be formalized in CL/GL
    \item Barr's Theorem guarantees then that classical results are provable in CL/GL
 \end{itemize}
\end{frame}

\begin{frame}
\frametitle{Algebraic theories in CL/GL}
 \begin{itemize}[<+->]   %  [<+-| alert@+>]
    \item Ring (commutative with unit): equational
    \item Local ring: $%\forall x.~
    \exists y.~(x\cdot y=1) \lor \exists z.~((1-x)\cdot z = 1)$
    \item Field: $%\forall x.~
    (x=0) \lor \exists y.~(x\cdot y=1)$ (makes $=$ decidable!)
    \item Alg.\ closed: $%\forall x.~
    \exists x.~x^{n+1}=a_0 + a_1 x + \cdots + a_n x^n $ ($n\in\nat$)
    \item Nilpotent $x$: $\bigvee_{n\in\nat}\, 0= x^{n+1}$

 \end{itemize}
\end{frame}

\begin{frame}
\frametitle{Hilbert's Nullstellensatz}
 \begin{itemize}[<+->]   %  [<+-| alert@+>]
    \item Consider fields $k \subset K$ with $K$ algebraically closed.
    Let $I$ be an ideal of $k[\vec x]$, and $V(I)$ the set of common zeros (Nullstellen)
    in $K$ of the polynomials in $I$. Then: for any $p\in k[\vec x]$ such that
    $p$ is zero on $V(I)$ there exists an $n$ such that $p^n \in I$.
    \item Hilbert's Nullstellensatz in its full generality is a strong classical theorem,
    with lots of special cases and variations
    \item Effective Nullstellensatz: \alert{compute} the $n$ such that $p^n \in I$
    \item Example: 
  $\mathbb{Q} \subset \mathbb{C}$, $I=(1+2x^2+x^4)$, $p= x-x^5$, $p^n \in I$~?

 \end{itemize}
\end{frame}

\end{document}
