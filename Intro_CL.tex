\documentclass[handout,11pt]{beamer}
%\documentclass[slides]{beamer}

%\usetheme{Warsaw}
\usetheme{Montpellier}
\usefonttheme[onlymath]{serif}
\setbeamertemplate{headline}{}
\setbeamertemplate{navigation symbols}{}

%\usepackage[english,norsk,german]{babel}
\usepackage[utf8x]{inputenc}
\usepackage{times}
\usepackage[T1]{fontenc}
\usepackage{proof}
\usepackage{url}
\usepackage{verbatim}
\usepackage{hyperref}
\usepackage{graphics}

\usepackage{color}
\newcommand{\blue}[1]{\textcolor{blue}{#1}}
\newcommand{\red}[1]{\textcolor{red}{#1}}
\newcommand{\green}[1]{\textcolor{green}{#1}}

%\usepackage[all]{xy}
\usepackage{amssymb,amsmath}


\newtheorem{proposition}{Proposition}

\newcommand{\weg}[1]{}
\newcommand{\Prop}{\mathit{Prop}}
\newcommand{\nat}{\mathbb{N}}
\newcommand{\set}[1]{\{#1\}}
\newcommand{\imp}{\rightarrow}
\newcommand{\liff}{\leftrightarrow}
%\newcommand{\proves}{\vdash}
\newcommand{\deca}{${\lor}{\exists}{\land}$}

\title{Coherent Logic --- an overview}
\author{Marc Bezem\\
Department of Informatics\\
University of Bergen
\mbox{ }\\
({\tt github.com/marcbezem/CL-PC22})
}

\date{September 2022}

\begin{document}

\frame{\titlepage}

\section[Outline]{}
\frame{\tableofcontents}


\section{Crash course in Coherent Logic (CL)}

\subsection{Basics}

\begin{frame}
\frametitle{Coherent logic preliminaries 1}
 \begin{itemize}[<+->]   %  [<+-| alert@+>]
    \item Fix a finite first-order signature $\Sigma$
    \item Positive formulas: built up from atoms using 
    $\top,\bot,\lor,\land,\exists$
    \item Coherent implications (sentences): $\forall\vec{x}.~(C \to D)$ 
    with $C,D$ positive formulas
    \item Coherent theory: axiomatized by coherent sentences
    \item \deca-formula: 
    $(\exists\vec{y}_1.A_1) \lor \cdots \lor (\exists\vec{y}_k.A_k)$,
    $k\geq 0$, with each $A_i$ a (possibly empty) conjunction of atoms
    \item Lemma 1. Every positive formula is (constructively) equivalent to
    a \deca-formula. Proof by induction:
    \begin{itemize}[<+->]   %  [<+-| alert@+>]
    \item Base cases: atom (one disjunct, empty $\exists$, one conjunct);\\ 
    $\bot$ (empty $\lor$); 
    $\top$ (one disjunct with empty $\exists,\land$)
    \item Induction cases: $\lor$ (trivial); 
    $\land$ (distributivity + $(\exists x.\varphi) \land (\exists y.\psi)$ iff 
                              $\exists xy.~(\varphi\land\psi)$);
    $\exists$ (commutes with $\vee$)    
    \end{itemize}
 \end{itemize}
\end{frame}

\begin{frame}
\frametitle{Coherent logic preliminaries 2}
 \begin{itemize}[<+->]   %  [<+-| alert@+>]
    \item Lemma 2. Every coherent implication is (constructively) 
    equivalent to a finite set of coherent implications
    $\forall\vec{x}.~(C \to D)$ 
    with $C$ a conjunction of atoms and $D$ a \deca-formula
    \item Proof. Use Lemma 1 to replace $C$ and $D$ by \deca-formulas.
    Then use $(\varphi\lor\psi)\to D$ iff $(\varphi\to D)\land(\psi\to D)$,
    and $(\exists y.\varphi)\to D$ iff $\forall y.~(\varphi\to D)$
    \item Notation: we use the format of Lemma 2, 
    leaving out the universal prefix, and omitting the premiss `$C\to{}$' 
    if $C\equiv\top$
    \item Discuss: $\exists y.\top$ and $\exists y.\bot$ and
    $\forall y.\top$ and $\forall y.\bot$
    \item Full compliance with Tarski semantics if $\Sigma$ has a constant
 \end{itemize}
\end{frame}

\begin{frame}
\frametitle{Examples}
 \begin{itemize}[<+->]   %  [<+-| alert@+>]
    \item all usual equality axioms, including congruence
    \item $p\lor np$ and $p\land np\to\bot$ 
    (NB $p\lor\neg p$ is \red{not} coherent)
    \item lattice theory: $%\forall x,y.~
    \exists z.~\mathit{meet}(x,y,z)$
    \item geometry: $%\forall x,y.~
    p(x)\land p(y) \to \exists z.~\ell(z) \land i(x,z) \land i(y,z)$
    \item rewriting, $\diamond$-property: $%\forall x,y,z.~
    r(x,y)\land r(x,z) \to \exists u.~r(y,u)\land r(z,u)$
    \item weak-tc-elim: $%\forall x,y.~
    r^*(x,y)\to (x=y)\lor\exists z.~r(x,z)\land r^*(z,y)$     
    \item seriality: $%\forall x.~
    \exists y.~s(x,y)$ (who needs a function?)
    \item field theory: $%\forall x.~
    (x=0) \lor \exists y.~(x\cdot y=1)$
    \item local ring: 
    $\exists y.~(x\cdot y = 1) \lor (\exists y.~((1-x)\cdot y = 1)$
    (equivalent to the more common: if $x+y$ is a unit, then $x$ is a
    unit or $y$ is a unit).
 \end{itemize}
\end{frame}


\begin{frame}
\frametitle{History of CL}
 \begin{itemize}[<+->]   %  [<+-| alert@+>]
    \item Skolem (1920s): coherent formulations of lattice theory 
    and projective geometry, calling the axioms ``Erzeugungsprinzipien"
    (production rules), anticipating ground forward reasoning. Using CL,
    \begin{itemize}[<+->]   %  [<+-| alert@+>]
    \item Skolem solved a decision problem in lattice theory
    \item Skolem gave a method to test in/dependence from the axioms
    of plane projective geometry (example: Desargues' Axiom)
    \end{itemize}
    
    \item Grothendieck (1960s): geometric morphisms preserve geometric
              logic (= coherent logic + infinitary disjunction).
    This is quite complicated, but we'll see a glimpse
    in the forcing semantics of coherent logic given later.
 \end{itemize}
\end{frame}

\subsection{Proof theory for CL}

\begin{frame}
\frametitle{A proof theory for CL}
 \begin{itemize}[<+->]   %  [<+-| alert@+>]
    \item In short: ground forward reasoning with case distinction and
    introduction of witnesses (ground tableau reasoning)
    \item In full: define inductively $\Gamma\vdash_{\vec{y}}^T A$, where 
    $A$ ($\Gamma$) atom (set of atoms) with all variables in $\vec{y}$, in case
    \begin{enumerate}
    \item[(base)] $A$ is in $\Gamma$, or in case 
    \item[(step)] $T$ has an axiom $\forall\vec{x}.~
    (C \to (\exists\vec{y}_1.B_1) \lor \cdots \lor (\exists\vec{y}_k.B_k))$
    such that for some sequence of terms $\vec{t}$ with variables in $\vec{y}$
    we have
    \begin{itemize}
    \item $C[\vec{t}/\vec{x}]$ is a subset of $\Gamma$, and
    \item $\Gamma,B_i[\vec{t}/\vec{x}]\vdash^T_{\vec{y},\vec{y_i}} A
          ~\text{for all $i = 1,\dots,n$}$ \quad
          (\red{NB $\vec{y_i}$ fresh wrt $\vec{y}$})
    \end{itemize}
    \end{enumerate}
    \item Rough visualization as a tree with inner nodes like 
    \[
\infer[\mathit{axiom}]{\Gamma}{\Gamma,B_1[\vec{t}/\vec{x}] \quad\cdots\quad \Gamma,B_n[\vec{t}/\vec{x}]}
    \]
    \item NB we omit conclusion $A$ in all the nodes, 
    but we should actually keep track of the $\vec{y},\vec{y_i}$.
    Looking ahead, pairs like $(\vec{y};\Gamma)$ will be
    the forcing conditions, $\approx$ finite Kripke worlds.
 \end{itemize}
\end{frame}


\begin{frame}\label{proofsearch}
\frametitle{Derivation trees in CL, example and general procedure}
 \begin{itemize}[<+->]   %  [<+-| alert@+>]   
    \item Let $T$ consists of $p\lor \exists x.~q(x)$ 
    and $p\to\bot$ and $q(y)\to r$
    \item Derivation tree for $\emptyset\vdash_{\emptyset}^T r$
    \[
\infer[p\vee \exists x.~q(x)]{\emptyset}
{
\infer[p\to\bot]{\set{p}}{(\bot)} & \infer[q(y)\to r]{\set{q(c)}}{\set{q(c),r}}
}
\]
    \item Tree construction: from $\emptyset$, repeat exhaustively 1,2,3 below
    \begin{enumerate}
    \item Pick a leaf node (${}\neq(\bot)$) without $A$ in its $\Gamma$ 
    (else done)
    \item Pick \red{fairly a $\Gamma$-false  instance} of an axiom of $T$
    (else fail: $\Gamma$ is a model of $T$ not containing $A$, 
    so $A$ is underivable)
    \item Extend the tree in the leaf node according to the instance  
    \end{enumerate}
    \item Fairness is tricky to define, but crucial for the following
    completeness result:
    \item The tree construction stops in 1 iff $A$ is derivable (\red{if-part!})
    \item Example for explaining un/fairness: $\exists y. s(x,y)$ and $p(0)$
 \end{itemize}
\end{frame}

\subsection{Metatheory}

\begin{frame}
\frametitle{Soundness and completeness wrt Tarski semantics}
 \begin{itemize}[<+->]   %  [<+-| alert@+>]
    \item Soundness easily proved by induction on $\Gamma\vdash_{\vec{y}}^T A$
    \item \red{Not} complete: $\emptyset\vdash_{\emptyset}^{\forall x.\bot} p$
    underivable without a constant in $\Sigma$
    \item Silly, let's assume a constant in $\Sigma$, or just $\exists x.\top$
    \item Proof of completeness: essentially non-constructive.
   %\item
    Assume $\forall\vec{y}.~(\Gamma \to A)$ holds in any model of $T$.
    Build the tree for $\Gamma\vdash_{\vec{y}}^T A$. Recall the that 
    the sets $\Gamma$ grow along the branches. If the tree is finite, 
    it is a proof (2 cannot happen). If not, it has an infinite branch
    by K\"onig's Lemma. Collect the set of variables $Y$ and the set
    of atoms $M$ along the infinite branch. Build a model with domain
    $\mathrm{Tm}^\Sigma(Y)$ and positive diagram $M$. This is a model of
    $T$ (\red{by fairness}) containing $\Gamma$ but not $A$. Contradiction.
 
    \item Proof theory easily extended to arbitrary coherent 
    conclusions of a coherent theory $T$.
 \end{itemize}
\end{frame}

\begin{frame}
\frametitle{Metatheoretic results and remarks}
 \begin{itemize}[<+->]   %  [<+-| alert@+>]
    \item Corollary of completeness: given a coherent theory $T$,
    classically provable coherent sentences are constructively provable
    \item For \alert{geometric} logic this is called Barr's Theorem
    \item Completeness and Barr's Theorem are \alert{not} constructive
    \item Barr's Theorem for \red{coherent} logic can be proved 
    constructively using a cut-elimination argument %(Coste \& Coste, Negri) 
    \item Coherent completeness wrt forcing semantics is constructively
    provable, but does not give the conservativity of classical reasoning
    \item NB: the forcing semantics is sound wrt intuitionistic logic 
    for arbitrary formulas
 \end{itemize}
\end{frame}

\subsection{Translation from FOL to CL}

\begin{frame}
\frametitle{Translation from FOL to CL}
 \begin{itemize}[<+->]
   \item Idea: introduce two new predicate symbols 
   $T(\psi),F(\psi)$ for each subformula $\psi$ of given $\varphi$, 
   with arity the number of free variables of $\psi$.
   The rules for signed tableaux are coherent axioms:
   \begin{itemize} 
   \item 
 if $\psi(\vec{x})\equiv\psi_1\wedge\psi_2$,
$\left\{\begin{array}{ll}
T(\psi)(\vec{x})\to T(\psi_1)(\vec{x})\wedge T(\psi_2)(\vec{x})\\
F(\psi)(\vec{x})\to F(\psi_1)(\vec{x})\vee F(\psi_2)(\vec{x})
\end{array}\right.$

\item

 if $\psi(\vec{x})\equiv\psi_1\vee \psi_2$ then ...
%$\left\{\begin{array}{ll}
%T(\psi)(\vec{x})\to T(\psi_1)(\vec{x})\vee T(\psi_2)(\vec{x})\\
%F(\psi)(\vec{x})\to F(\psi_1)(\vec{x})\wedge F(\psi_2)(\vec{x})
%\end{array}\right.$

\item
  if $\psi(\vec{x})\equiv\psi_1\to \psi_2$ then 
$\left\{\begin{array}{ll}
T(\psi)(\vec{x})\to F(\psi_1)(\vec{x})\vee T(\psi_2)(\vec{x})\\
F(\psi)(\vec{x})\to T(\psi_1)(\vec{x})\wedge F(\psi_2)(\vec{x})
\end{array}\right.$

\item

 if $\psi(\vec{x})\equiv\neg \psi_1$ then 
$\left\{\begin{array}{ll}
T(\psi)(\vec{x})\to F(\psi _1)(\vec{x})\\
F(\psi)(\vec{x})\to T(\psi _1)(\vec{x})
\end{array}\right.$

\item

 if $\psi(\vec{x})\equiv\forall y.\psi_1(\vec{x},y)$ then 
$\left\{\begin{array}{ll}
T(\psi)(\vec{x})\to T(\psi_1)(\vec{x},\red{y})\\
F(\psi)(\vec{x})\to \exists y.F(\psi_1)(\vec{x},y)
\end{array}\right.$

\item

 if $\psi(\vec{x})\equiv\exists y.\psi_1(\vec{x},y)$ then 
$\left\{\begin{array}{ll}
T(\psi)(\vec{x})\to \exists y.T(\psi_1)(\vec{x},y)\\
F(\psi)(\vec{x})\to F(\psi_1)(\vec{x},\red{y})
\end{array}\right.$

\item
 if $\psi(\vec{x})$ is atomic then 
 $(T(\psi)(\vec{x})\wedge F(\psi)(\vec{x})) \to \bot$
 \end{itemize} 
 \item By the completeness of tableaux: $\varphi$ is a tautology iff
 $F(\varphi)\vdash^{Coh(\varphi)}_\emptyset \bot$, 
 with $Coh(\varphi)$ all the above axioms
 \end{itemize} 
\end{frame}

\begin{frame}
\frametitle{Example in propositional logic: Peirce's Law}
 \begin{itemize}[<+->] 
    \item Peirce's Law: $\varphi\equiv((p\imp q)\imp p)\imp p$
    \item To prove: $F(((p\imp q)\imp p)\imp p) 
    \vdash^{Coh(\varphi)}_\emptyset\bot$    
    \item Part of $Coh(\varphi)$ that is actually used:
    \begin{enumerate}[<+->]
    	%\item $F(((p\imp q)\imp p)\imp p)$
    	\item $F(((p\imp q)\imp p)\imp p) \imp (T((p\imp q)\imp p) \wedge F(p))$
    	\item $T((p\imp q)\imp p) \imp (F(p\imp q) \vee T(p))$
    	\item $F(p\imp q) \imp (T(p) \wedge F(q))$
    	\item $(T(p) \wedge F(p)) \imp \bot$
    \end{enumerate}
    \item Proof: use 1, 2, 3 and split on $F(p\imp q) \vee T(p)$, ...
    \item Details on the blackboard
    \item Proof of $\varphi$: take $T\equiv\lambda\varphi.\,\varphi$,
     $F\equiv\lambda\varphi.\,\neg\varphi$. Then 1,2,3,4 are easy
     (but classical), the CL proof is also a proof 
     in propositional logic, and we finish by RAA
 \end{itemize}
\end{frame}

\begin{frame}
\frametitle{Example in predicate logic: the Drinker's Paradox}
 \begin{itemize}[<+->] 
    \item Drinker's Paradox: $\varphi\equiv\exists x.~(d(x)\to \forall y.d(y))$
    \item To prove: $F(\exists x.~d(x)\to \forall y.d(y))) 
    \vdash^{Coh(\varphi)}_\emptyset\bot$    
    \item Part of $Coh(\varphi)$ that is actually used:
    \begin{enumerate}[<+->]
    	%\item $F(\exists x.~d(x)\to \forall y.d(y)))$
    	\item \red{no} take-off without $\exists x. \top$, 
    	alternative: prove $F(\varphi)\vdash^{Coh(\varphi)}_{\set{\red{c}}}\bot$
    	\item $\forall\red{x}.~( F(\exists x.~d(x)\to \forall y.d(y))) \imp 
    	F(d(\red{x})\to \forall y.d(y)))$
    	\item $\forall\red{x}.~(F(d(\red{x})\to \forall y.d(y)) \imp 
    	(T(d(\red{x})) \wedge F(\forall y.d(y))))$
    	\item $F(\forall y.d(y)) \imp \exists y. F(d(y))$
    	\item $\forall x.~(T(d(x)) \wedge F(d(x))) \imp \bot$   	
    \end{enumerate}
    \item Proof: use 1 and get $c$, instantiate 2 and 3 with $c$
    and get $T(d(c)) \wedge F(\forall y.d(y))$, so by 4 we get $c'$
    with $F(d(c'))$, ...
    \item Details on the blackboard    
    \item Proof of $\varphi$ in FOL: take $T\equiv\lambda\varphi.\,\varphi$,
     $F\equiv\lambda\varphi.\,\neg\varphi$. Then 1--5 are easy
     (Tarski and classical), the CL proof is also a proof in FOL,
     and we finish by RAA
 \end{itemize}
\end{frame}


\begin{frame}
\frametitle{Translation from FOL to CL (ctnd)}
 \begin{itemize}[<+->]
   \item Skolem (\red{1920}): Every FOL theory has a definitional 
   extension that is equivalent to a $\forall\exists$ theory
   \item Many variations possible (Polonsky, Dyckhoff \& Negri, Fisher, Mints)
   \item Possible objectives: fewer new predicates, fewer CL axioms ...,
   keeping a coherent axiom coherent
   \item Polonsky proposed several improvements, starting from NNF,
   flipping polarities, also using reversed tableaux rules
   \item Dyckhoff \& Negri: add $T(\psi)(\vec{x}) \to \psi(\vec{x})$ 
   and $(F(\psi)(\vec{x})\wedge \psi(\vec{x})) \to \bot$
   for all atomic $\psi$ and obtain:
   Every FOL theory has a positive semi-definitional extension 
   that is equivalent to a CL theory
   % new P such that P(x) -> \phi(x), examples above, D&N (xiv),(xv)
   \item Consequences in CL are always constructive
   \item Translation of FOL to CL contains many non-constructive steps,
   often more than necessary
 \end{itemize}
\end{frame}

\subsection{Evaluation of CL as a fragment of FOL}

\begin{frame}
\frametitle{Evaluation of CL as a fragment of FOL}
 \begin{itemize}[<+->]   %  [<+-| alert@+>]
  \item Constructive, with classical logic a conservative extension
  \item Simpler metatheory: proof theory, completeness, 
    conservativity of skolemization (elimination of $\exists$)
  \item Applications to metamathematics: independence, decision problems
  \item Other applications:
   \begin{itemize}[<+->]   %  [<+-| alert@+>]
    \item automated reasoning, supporting proof assistants 
    \item model finding
    \item constructive algebra
   \end{itemize}
 \end{itemize}
\end{frame}


\section{Automated reasoning in CL}

\subsection{Automated reasoning}

\begin{frame}
\frametitle{Automated reasoning (AR)}
 \begin{itemize}[<+->]   %  [<+-| alert@+>]
  \item We focus on AR in (fragments of) FOL
  \item There are dozens of FOL provers (Vampire wins CASC)
  \item TPTP is a large database of AR problems (CNF/FOL/HOL)
  \item Different branch of AR: model finding (SAT/CNF/FNT) 
  \item There are a handful of CL provers (competitive on
  CL problems, but not on FOL problems):
  \begin{itemize}[<+->]
   \item SATCHMO+ (Bry et al.)
   \item Argo, Larus (Janicic et al.)   
   \item Geo (Nivelle et al.)
   \item Colog (Fisher)   
   \item EYE (De Roo, semantic web) 
  \end{itemize}
  \item Most CL provers support only 0-ary function symbols
  \item We describe later how to eliminate function symbols
    %\item Janicic, Kordic (90s): geometry prover Euclid
    %\item Abdennadher, Sch\"utz (90s): \alert CPUHR (\alert Constraints)
    %\item B, Coquand, Fisher, Nivelle (00s): geometric/coh.\ logic
 \end{itemize}
\end{frame}

\begin{frame}
\frametitle{Rationale for automated reasoning in CL}
 \begin{itemize}[<+->] %[<+->]   %  [<+-| alert@+>]
    \item Expressivity of CL is between CNF (resolution) and FOL
    \item Different balance: expressivity versus efficiency
    \item Skolemization (elimination of $\exists$) not necessary
    \begin{itemize}
       \item Skolemization makes the Herbrand Universe infinite
       \item Why skolemize an axiom like $p(x,y)\to\exists z.~p(x,z)$?
       \item Skolemization changes meaning (problematic for interactive
       theorem proving, and for obtaining proof objects)
       \item Skolem functions obfuscate symmetries (cf.\ $\diamond$-property)
       \item \red{But}: skolemized proofs can be exponentially shorter!
    \end{itemize}
   \item Ground forward reasoning is very simple and intuitive,
   proof objects can easily be obtained
   \item \red{But}: non-ground proofs can be exponentially shorter!   
 \end{itemize}
\end{frame}



 
\subsection{Elimination of function symbols}

\begin{frame}
\frametitle{Elimination of function symbols}
 \begin{itemize}[<+->]
  \item Idea: use the graph instead of the function, i.e.,
  new $(n{+}1)$-predicates for $n$-ary functions, for example:
   \begin{itemize}[<+->]
    \item For constants: $c(x)$ (expressing $c=x$), 
     axiom $\exists x.\,c(x)$
    \item For unary functions: 
    $s(x,y)$ (expressing $s(x)=y$), axiom $\exists y.\,s(x,y)$
   \end{itemize} 
   \item Example: the term $f(s(x),o)$ leads to a condition
   $s(x,y)\land o(u)\land f(y,u,z)$ after which every occurrence of
   $f(s(x),o)$ is replaced by $z$. Then $\forall\vec{x}.~(C \to D)$ becomes 
   $\forall x,y,u,z,\vec{x}~(s(x,y)\land o(u)\land f(y,u,z)\land C'\to D')$ 
   where $C',D'$ are the result of the substitution in $C$ and $D$.
   \item Example: $a=b$ becomes $a(x) \land b(y) \to x=y$
   \item Unicity, e.g., $c(x)\land c(y)\to x=y$, \alert{not} required!
   (since the new conditions only occur in negative positions)
 \end{itemize}
\end{frame}

\begin{frame}
\frametitle{Puzzle, formalized in CL with functions (Nivelle)} 
 \begin{itemize}[<+->]
     \item Can one color each
     $n\in\nat$ either red or blue but not both such that, 
     if $\red{n}$ is red, then $\blue{n{+}2}$ is blue, 
     and if  $\blue{n}$ is blue, then $\red{n{+}1}$ is red?
    \item No: consider $\red{0}?\blue{2}\red{3}\ldots$
    and $\blue{0}\red{1}?\blue{3}\red{4}\ldots$
    \item CL theory:
    \begin{enumerate}
       \item $r(x) \vee g(x)$ 
       \item $r(x) \wedge g(x) \imp \bot$
       \item $r(x) \imp g(s(s(x)))$ 
       \item $g(x) \imp r(s(x))$
     \end{enumerate}
   \item Do we miss something?
   \item Yes, domain non-empty: 
    \begin{enumerate}       
       \setcounter{enumi}{4}       
       \item $\exists x.~\top$
    \end{enumerate}
 \end{itemize}
\end{frame}

\begin{frame}
\frametitle{Puzzle, function eliminated}
 \begin{itemize}[<+->]
    \item Living dangerously: demo of {\tt hdn.co} in Colog (Fisher, '12)
    \item Just in case the demo fails: refutation on blackboard
    \begin{enumerate}
      \item $r(x) \vee g(x)$ 
      \item $r(x) \wedge g(x) \imp \bot$
      \item $r(x) \wedge s(x,y) \wedge s(y,z) \imp g(z)$ 
      \item $g(x) \wedge s(x,y) \imp r(y)$    
      \item $\exists x.~\top$
      \item $\exists y.\,s(x,y)$
  \end{enumerate}
  \item Solution of puzzle before eliminating 
  the function:
    \begin{itemize}
    \item Note that the substitution principle is valid
    \item Substitute $(s(x) = y)$ for $s(x,y)$ in 3,4,6:
     \begin{itemize}
     \item Regarding 6, $\exists y.\,s(x)=y$ is trivial   
     \item Regarding 4, $g(x) \wedge s(x)=y \imp r(y)$ is 
     equivalent to $g(x) \imp r(s(x))$
     \item Similarly for 3 (and, in general, for any function)   
     \end{itemize}
    \end{itemize}
  \end{itemize}
\end{frame}

\subsection{Proof search strategies}

\begin{frame}
\frametitle{Depth-first proof search in CL}
 \begin{itemize}[<+->]   %  [<+-| alert@+>]
    \item Recall the search procedure on slide \ref{proofsearch}
    \item Any open leaf is fine, so we always take the leftmost
    \item What instance of which $\Gamma$-false axiom to pick?
    \item NB two trees: derivation tree and the search space
    organized as a tree
    \item Depth-first search: pick always the first $\Gamma$-false 
    axiom from the list, and use the `simplest' (`oldest') instance
    \item Obviously incomplete, but often OK with
    favourable ordering of coherent axioms:
    \begin{enumerate}
       \item Facts first, then Horn clauses ($\to$ goal first)
       \item Disjunctive clauses (cause branching)
       \item Existential axioms (cause new variables)
       \item Disjunctive existential axioms (cause both, the worst)
    \end{enumerate}
    \item Example: $\exists y.\,s(x,y)$ should not be put first!
 \end{itemize}
\end{frame}


\begin{frame}
%\frametitle{Example: Diamond property closed under reflexive closure}
  \begin{columns}
    \column{\dimexpr\paperwidth-10pt}
    \small\verbatiminput{LABresources/dpe.in}
  \end{columns}
\end{frame}


\begin{frame}
\frametitle{Breadth-first proof search in CL}
 \begin{itemize}[<+->]   %  [<+-| alert@+>]
  \item Recall: $\Gamma$ is the condition of the leaf node at hand 
  \item Breadth-first search: collect all `simplest' instances  
  of $\Gamma$-false axioms and use them exhaustively
  \item Breadth-first search: complete, but often infeasible
  \item With only constants, depth-first complete for forms 1 and 2
  \item Depth-first search not complete for \red{one single existential} clause, subtle:
\texttt{  p(a). p(b). q(b) -> goal. p(X),p(Y) -> dom(U),p(U),q(X),r(Y).}
    \item Wanted: fair application of axioms of form 3 and 4 
    \item Cycling depth-first: depth-first for forms 1 and 2, plus cycling
    through the (disjunctive) existential clauses, using instances
    with the `oldest' constants first. Complete.
 \end{itemize}
\end{frame}



\begin{frame}
\frametitle{Automated reasoning in CL, conclusions}
 \begin{itemize}[<+->]
   \item Good start: Newman's Lemma (Bezem \& Coquand,'03)
   \item Limited success in CASC: 50\% in FOF (Geo, Nivelle'06)
   \item Readable proofs can be extracted from CL proofs
   \item Highlight: Hessenberg's Theorem (B, Hendriks, JAR'08)
   \item Promising: using SAT techniques (Janicic et al.)
   %\item Colog:\\\url{www.csupomona.edu/~jrfisher/colog2012/}
   \item A case study, if time allows: Newman's Lemma, stating
   that, for any strongly terminating relation $r(x,y)$, if
   $r$ is locally confluent, then $r$ is confluent.
   Informal proof on blackboard, code in \texttt{nl.in}.
   Many interesting aspects.
 \end{itemize}
\end{frame}


\section{Applications of CL}

\subsection{Proof assistants}

\begin{frame}
\frametitle{Proof assistants}
 \begin{itemize}[<+->]   %  [<+-| alert@+>]
    \item In proof assistants, proof objects are required
    \item CL proofs are readable and easily convertable
    \item Provers outputting proof objects:
    \begin{itemize}
       \item \texttt{cl.pl} (B, exports proofs to Coq, also used
       to verify them)
       \item \url{coherent} (Isabelle tactic, Berghofer)
       \item ArgoCLP (Coq, Isar, natural language)
    \end{itemize}
    \item Modern automated support of proof assistants centers
    around specialized tools for decidable fragments of FOL,
    using SAT Modulo Theories-techniques. Very useful is, e.g.,
    the tactic \texttt{lia} (linear arithmetic) in Coq.   
 \end{itemize}
\end{frame}

\subsection{Model finding}

\begin{frame}
\frametitle{Model finding}
 \begin{itemize}[<+->]   %  [<+-| alert@+>]
    \item Satisfiability in FOL is co-RE, so restrict to finite models
    \item Naive approach: try to find a model with 1 element,
    then with 2 elements, and so on. Quantifiers $\forall,\exists$
    are written out (`grounding'), and the resulting (rapidly growing)
    propositions are fed to a SAT-solver
    \item Many clever tricks can actually make this to work
    \item CL proof search is not finite model complete: $\exists y.~s(x,y)$
    \item Solution (Nivelle): use (exhaustively) old constants 
    before you generate a new one + use lemma learning
    %\item Complete, but lemma learning \red{absolutely necessary}!
    \item Success in CASC'07: 81\% in FNT (Geo, Nivelle)\\
             (Paradox, based on Minisat, winner with 85\%)
    \item CL competitive on problems `too big to ground'
    %\item Example: formal verification of a Kripke model for simplicial sets (B \& Coquand, TCS)
 \end{itemize}
\end{frame}              

\subsection{Constructive algebra}

\begin{frame}
\frametitle{Constructive algebra}
 \begin{itemize}[<+->]   %  [<+-| alert@+>]
    \item Pioneers of applying CL/GL to constructve algebra:\\
  Coste, Lombardi, Roy, %(Effective Nullstellens\"atze ...), 
Coquand
    \item Idea: making constructive sense of classical proofs by
  exploiting that significant parts of algebra can be formalized in CL/GL
    \item Barr's Theorem guarantees then that classical results are provable in CL/GL
 \end{itemize}
\end{frame}

\begin{frame}
\frametitle{Algebraic theories in CL/GL}
 \begin{itemize}[<+->]   %  [<+-| alert@+>]
    \item Ring (commutative with $1\neq0$): equational axioms
    \item Local ring: $%\forall x.~
    \exists y.~(x\cdot y=1) \lor \exists y.~((1-x)\cdot y = 1)$
    \item Field: $%\forall x.~
    (x=0) \lor \exists y.~(x\cdot y=1)$ (makes $=$ decidable!)
    \item Alg.\ closed: $%\forall x.~
    \exists x.~x^{n+1}=a_0 + a_1 x + \cdots + a_n x^n $ (all $n\in\nat$),
    so infinitely many coherent axioms
    \item Positive formula using an infinite disjunction:
    $\bigvee_{n\in\nat}\, 0= x^{n+1}$, expressing that $x$ is nilpotent

 \end{itemize}
\end{frame}

\begin{frame}
\frametitle{Hilbert's Nullstellensatz}
 \begin{itemize}[<+->]   %  [<+-| alert@+>]
    \item Consider fields $k \subset K$ with $K$ algebraically closed.
    Let $I$ be an ideal of $k[\vec x]$, and $V(I)$ the set of common zeros (Nullstellen)
    in $K$ of the polynomials in $I$. Then: for any $p\in k[\vec x]$ such that
    $p$ is zero on $V(I)$ there exists an $n$ such that $p^n \in I$.
    \item Example: 
  $\mathbb{Q} \subset \mathbb{C}$, $I=(x^4+2x^2+1)$, $p= x^5-x$, $p^2 \in I$
    \item In its full generality, Hilbert's Nullstellensatz is a 
    strong classical theorem, with lots of special cases and variations
    \item Effective Nullstellensatz: \alert{compute} the $n$ such that $p^n \in I$
    \item Dynamical method in algebra: Effective Nullstellensätze,
    Coste, Lombardi, Roy, 2001 (Dynamic method = CL proof)

 \end{itemize}
\end{frame}

\end{document}


Betal papers

Geometric resolution: A proof procedure based on finite model search
H Nivelle, J Meng - International Joint Conference on Automated …, 2006 - Springer

Th{\'e}ories coh{\'e}rentes et topos coh{\'e}rents,
    M.-F.~Coste and M.~Coste, 1975.


\begin{frame}
\frametitle{Title}
 \begin{itemize}[<+->]
    \item 
 \end{itemize}
\end{frame}



